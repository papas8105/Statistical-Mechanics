\documentclass[12pt]{report}
%\author{George}
\usepackage{geometry}
\geometry{a4paper,total={210mm,297mm},
 left=20mm,
 right=20mm,
 top=20mm,
 bottom=20mm,
 }
\usepackage{kerkis}
\pagenumbering{roman}
\usepackage{fontspec}
\usepackage{xunicode}
\usepackage{xltxtra}
\usepackage{xgreek}
\defaultfontfeatures{Ligatures=TeX,Numbers=OldStyle ,Scale=MatchLowercase} 
\setmainfont[Mapping=tex-text]{Times New Roman}
\setsansfont{Gentium}
\usepackage{parskip}
\usepackage{listings}
\usepackage[usenames,dvipsnames,svgnames,table]{xcolor}
\usepackage{pstricks-add}
\definecolor{light-gray}{gray}{0.95}
\usepackage{listings}
\lstset{
    basicstyle=\footnotesize\ttfamily,
    escapechar=¢,
    language=python,
    frame=single,
    frameround=tttt,
    showstringspaces=false,
    backgroundcolor=\color{light-gray}
}
\newcommand*{\ipythonprompt}[1]{\makebox[0pt][r]{\textbf{In [#1]:}\hspace{1em}}}
\usepackage{graphics}
\newcommand{\ttt}{\texttt}
\begin{document}
\begin{center}
 \Large Spotting The Correct Algorithm
\end{center}

\textbf{1.} Consider a one dimensional system of N sites with periodic boundary conditions, with a single particle hopping between 
neighboring sites at time $t = 0,1,2,\ldots$ as shown in the next figure. A move from site $i$ to site $i+1$ modulo N is 
"to the right" and a move from site $i$ to site $i-1$ is "to the left". 
\[
\psset{xunit=0.75cm,yunit=0.75cm,algebraic=true,dotstyle=o,dotsize=3pt 0,linewidth=0.8pt,arrowsize=3pt 2,arrowinset=0.25}
\begin{pspicture*}(-3.82,-2.88)(3.35,3.12)
\pscircle[linewidth=2pt](-0.24,0.04){2.05}
\rput[tl](2.54,1.72){$$ i $$}
\rput[tl](1.03,3.0){$$ i + 1 $$}
\begin{scriptsize}
\psdots[dotsize=5pt 0,dotstyle=*,linecolor=red](0.78,2.58)
\psdots[dotsize=5pt 0,dotstyle=*,linecolor=red](-1.31,2.56)
\psdots[dotsize=5pt 0,dotstyle=*,linecolor=red](-2.78,1.07)
\psdots[dotsize=5pt 0,dotstyle=*,linecolor=red](-2.76,-1.03)
\psdots[dotsize=5pt 0,dotstyle=*,linecolor=red](-1.26,-2.49)
\psdots[dotsize=5pt 0,dotstyle=*,linecolor=red](0.83,-2.48)
\psdots[dotsize=5pt 0,dotstyle=*,linecolor=red](2.3,-0.98)
\psdots[dotsize=5pt 0,dotstyle=*,linecolor=red](2.28,1.11)
\end{scriptsize}
\end{pspicture*}
\]
The two following programs implement the \textbf{Markov-chain Monte Carlo} algorithm such at that each 
time step, the particle moves with probability 1/2 to the right and with probability 1/2 to the left. 
We also have to note that the algorithm (in both programs) satisfies detailed balance with a constant 
probability on all sites and that it is irreducible thus it cannot be broken up into two independent 
processes.
\lstinputlisting[language = Python]{1.py}
\lstinputlisting[language = Python]{2.py}
\end{document}
